\section{Conclusion}
With the popularity of hybrid cloud systems, the number of platforms trying to centralise their management and the billing system is continuously growing. This Cloud management platform differs depending on the software, and each of them is suitable for specific purposes. A lot of works that compare a separate platform functionality are written and published, which indicates the relevance of this topic. In this paper, we examined the possibility of centralised and standardised testing of CMPs based on web platforms, local docker platforms and libraries.

As part of this work written a test environment and created an architecture for multi-platform testing that systematised comparisons and provides the ability to run all tests with just one starting file, which in the future can be used by other researchers to validate the experiments. The architecture is modular and very flexible, which provides the possibility of its constant expansion. The raw result stored in text form, which in the future by the same test-bed generates not only single and combined graphs, but also tables in latex format.

The aim of the work was not to compare the platforms but to create a test environment and, as an example used libcloud (library), mistio(docker composed containers), ManageIQ(single image container), CloudcheckR(website with open API). The results are entirely consistent since libraries are the easiest way to manage platforms and they do not have an overhead, and they showed the best results. In the evaluation of provider management between two local running platforms mistio showed itself better since ManageIQ has the much more full range of functionality and possibilities leading to more significant overhead.
All the data and findings published in opensource to keep the studying reusable and repeatable.